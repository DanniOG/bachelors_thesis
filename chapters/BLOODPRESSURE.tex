%!TEX ROOT=main.tex
\chapter{Hemodynamické parametry}
Tato kapitola se zaměří na popis krevního tlaku, centrálního aortálního tlaku, středního arteriálního tlaku, rychlosti šíření pulzní vlny, metod měření krevního tlaku a na rešerši přístrojů pro měření hemodynamických parametrů krevního řečiště určovaných neinvazivně z tvaru tlakové křivky.
\section{Krevní tlak}
Krevní tlak je veličina, která vyjadřuje velikost síly proudící krve působící na stěnu cévy. Velikost krevního tlaku závisí zejména na síle kontrakce srdce, objemu krve v krevním řečišti a
poddajnosti cév. Krevní tlak se nejčastěji uvádí pomocí dvou parametrů. První, systolický vyjadřuje maximální tlak v průběhu srdečního cyklu
a druhý, diastolický vyjadřuje minimální hodnotu tlaku v průběhu srdečního cyklu. Krevní tlak se uvádí v jednotkách milimetrů rtuti ($mmHg$), což odpovídá $133.322 \ Pa$. Za optimální lidský krevní tlak se považují hodnoty do $120 \ mmHg$ systolického a do $80 \ mmHg$ diastolického tlaku.
\cite{cite:BP}

\subsection{Centrální aortální tlak}
Centrální aortální tlak je tlak v aortě, do které krev putuje při kontrakci srdce.
Tento tlak je možné odhadnout z hodnot brachiálního nebo radiálního tlaku a tvaru tlakových pulzací měřených manžetou, která je umístěna na paži nebo zápěstí.\cite{cite:CBP}

\subsection{Střední arteriální tlak}
Střední arteriální tlak (MAP) je průměrná hodnota krevního tlaku v jednom srdečním cyklu.
\begin{figure}[H]
    \caption{Průběh tlakové křivky pro výpočet středního arteriálního tlaku \cite{cite:4}}
    \includegraphics[width=1\textwidth]{pictures/map.jpg}
\end{figure}
Rovnice pro přesný výpočet je
\begin{equation}
    MAP = \frac{1}{T} \int_{t}^{t + T} BP(\tau) d\tau
\end{equation}
Rovnice pro aproximaci je
\begin{equation}
    MAP = DP + \frac{1}{3}(SP - DP)
\end{equation}
Kde $DP$ je diastolický tlak a $SP$ je systolický tlak.
\section{Metody měření krevního tlaku}
Metody měření tlaku můžou být invazivní nebo neinvazivní, manuální či automatizované. Jednou z nejčastěji používaných metod pro neinvazivní měření
krevního tlaku je auskultační metoda, která používá rtuťového sphygmomanometru a stetoskopu pro poslech Korotkovových zvuků. Další často používanou metodou je oscilometrická metoda.
\cite{cite:Fabian}
\subsection{Oscilometrická metoda}
Metoda spočívá v měření objemových pulzací v tepnách přenášených se přes manžetu do přístroje, ve kterém se vyhodnocují.
Amplituda těchto pulzací je závislá na rozdílu tlaku uvnitř a vně tepny, tzv. transmurální tlak, kdy největší amplituda je při nulovém transmurální tlaku, to je při středním arteriálním tlaku.
\cite{cite:Fabian} Oscilometrická metoda měření tlaku využívá navrhovaný systém CarDi.
\begin{figure}[H]
    \caption{Graf oscilometrických pulzací \cite{cite:Fabian}}
    \includegraphics[width=1\textwidth]{pictures/oscilometricky_tlak.jpg}
\end{figure}
Oscilometrická metoda měření tlaku využívá navrhovaný systém CarDi.
\raggedbottom
\section{Rychlost šíření pulzní vlny}
Rychlost šíření pulzní vlny (PWV) je rychlost, během systolické kontrakce srdce, jakou se tlaková pulzní vlna krve propaguje tepnami. Parametr PWV je jedním ze základních ukazatelů tuhosti cév, tj. rozvoje aterosklerózy.
Čím jsou cévy méně poddajné, tzn. více tuhé, tím je hodnota PWV větší.
\begin{figure}[H]
    \caption{Postup odražené krevní tlakové vlny v těle \cite{cite:5}}
    \includegraphics[width=1\textwidth]{pictures/pwv_body.jpg}
\end{figure}
Jeden ze způsobů určení parametru PWV je poměr dvojnásobné vzdálenosti od hrudního zářezu ke stydké kosti $l$ a rozdíl času primární tlakové vlny $t_1$ a odražené tlakové vlny $t_2$.
\begin{equation} \label{eq:pwv}
    PWV = \frac{2l}{t_2 - t_1}
\end{equation}
\begin{figure}[H]
    \caption{Krevní tlaková vlna s vyznačenými parametry pro výpočet parametru PWV. \cite{cite:7}}
    \includegraphics[width=1\textwidth]{pictures/pwv_pressure_wave.jpg}
\end{figure}
Kritéria pro PWV u člověka jsou
\begin{table}[H]
    \label{tab:pwv_criteria}
    \caption{Kritéria rychlosti pulzní vlny u člověka \cite{cite:Fabian}}
    \begin{ctucolortab}
        \begin{tabular}{ccc}
            \toprule
            PWV                & Jednotky      & Stav           \\ \midrule
            PWV < 7            &               & Optimální      \\
            7 $\leq$ PWV < 10  & $\frac{m}{s}$ & Normální       \\
            10 $\leq$ PWV < 12 &               & Zvýšené riziko \\
            12  $\leq$ PWV     &               & Abnormální     \\

            \bottomrule
        \end{tabular}
    \end{ctucolortab}
\end{table}
\section{Rešerše přístrojů pro měření hemodynamických parametrů krevního řečiště}
Tato sekce se zaměří na rešerši přístrojů pro měření hemodynamických parametrů krevního řečiště určovaných neinvazivně z tvaru
tlakové křivky.
\subsection{SphygmoCor XCEL PWA/PWV}
SphygmoCor XCEL PWA/PWV je automatický systém pro analýzu krevní tlakové vlny pomocí okulzivní manžety.
\cite{cite:SphygmoCor}
\begin{figure}[H]
    \caption{SphygmoCor XCEL PWA/PWV \cite{cite:SphygmoCor}}
    \includegraphics[width=0.8\textwidth]{pictures/XCEL_System.jpg}
\end{figure}
Mezi měřené parametry patří centrální aortální tlak, augmentační index a PWV. Pro měření parametru PWV je potřeba přidaná manžeta na stehno.
\cite{cite:SphygmoCor}
\par
Ovládání přístroje je pomocí připojeného osobního počítače přes USB s nainstalovaným softwarem od firmy SphygmoCor. Software spouští terapii a následně i vyhodnocuje naměřené hodnoty. Samostatný přístroj bez osobního počítače a dedikovaného softwaru není schopný spustit terapii.
\cite{cite:SphygmoCor}
\subsection{Uscom BP+}
Uscom BP+ je automatizovaný systém pro měření centrálního aortálního tlaku, augmentačního indexu a analýzy křivky krevního tlaku a dalších parametrů pomocí pažní manžety při suprasystolickém tlaku.
Pro použití a operaci přístroje není zapotřebí speciální školení uživatelům a najde uplatnění při hypertenzi, srdečním selhání, intenzivní péči a všeobecné praxi
\cite{cite:Uscom}
\begin{figure}[H]
    \caption{Uscom BP+ \cite{cite:Uscom}}
    \includegraphics[width=0.8\textwidth]{pictures/uscom_bp.jpg}
\end{figure}
Systém BP+ provádí terapii a analýzu naměřených hodnot v jednom systému tj. bez potřeby nadřazeného systému.
\cite{cite:Uscom}
\subsection{Arteriograph}
Arteriograph (Tensiomed) je systém pro neinvazivní měření hemodynamických parametrů krevního řečiště oscilometrickou metodou a pomocí jedné pažní manžety.
Metoda měření je patentována (US Pat. No. 20070106162) a validována invazivně.
\begin{figure}[H]
    \caption{Tensiomed Arteriograph \cite{cite:Arteriograph}}
    \includegraphics[width=0.8\textwidth]{pictures/arteriograph.jpg}
\end{figure}
Arteriograph může fungovat samostatně pro měření pouze krevního tlaku. Přístup k měření dalších hemodynamických parametrů je potřeba systém připojit k nadřazenému systému pomocí bluetooth s nainstalovaným specializovaným softwarem od Tensiomed.
Po připojení k nadřazenému systému, Arteriograph zasílá naměřené data během terapie v reálném čase, kde se naměřené výsledky zpracují a zobrazí.
\cite{cite:Arteriograph}

\subsection{Fukuda Denshi VaSera VS-1500N}
VaSera VS-1500N je diagnostický přístroj měřící stav cévního systému a provádějící screening aterosklerózy. Přístroj pracuje na principu měření pulzové tlakové vlny. Data jsou snímána ze čtyř tlakových manžet z končetin, dvou EKG elektrod a mikrofonu zaznamenávajícího srdeční ozvy.
Mezi měřené parametry patří srdeční frekvence, tlak ve všech čtyřech končetinách, preejekční periodu, ejekční čas, čas dosažení maximálního tlaku a určuje biologický věk cév.\cite{cite:Vasera}
\begin{figure}[H]
    \caption{Fukuda Denshi VaSera VS-1500N \cite{cite:Vasera}}
    \includegraphics[width=0.8\textwidth]{pictures/vs_1500n.jpg}
\end{figure}
Mezi dopočítávané parametry patří rychlost šíření pulzní vlny, CAVI, ABI a další. Možnost měření PWV, EKG je možné, až po připojení volitelných modulů.\cite{cite:Vasera}
\par
Systém se dá použít samostatně bez nadřazeného systému, ale může se připojit k softwaru VSS-10, který umožňuje archivaci pacientský dat, porovnávání a export dat na paměťovou kartu Compact Flash. Uvnitř systému je
vestavěná barevná tiskárna zapisující na termopapír o šířce 145 mm.\cite{cite:Vasera}
