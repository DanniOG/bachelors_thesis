\documentclass{ctuthesis}

\ctusetup{
	%preprint = \ctuverlog,
%	mainlanguage = english,
%	titlelanguage = czech,
	mainlanguage = czech,
	otherlanguages = {slovak,english},
	title-czech = {Systém pro neinvazivní měření hemodynamických parametrů},
	title-english = {System for non-invasive measurement of hemodynamic parameters},
	%subtitle-czech = {Cesta do tajů kdovíčeho},
	%subtitle-english = {Journey to the who-knows-what wondeland},
	doctype = B,
	faculty = F3,
	department-czech = {Katedra teorie obvodů},
	department-english = {Department of circuit theory},
	author = {Daniel Klamrt},
	supervisor = {Ing. Vratislav Fabián, Ph.D},
	supervisor-address = {Praha, Technická 1902/2, \\ místnost: B2-36},
	supervisor-specialist = {Ing. Yegor Mikhailov},
	fieldofstudy-english = {Medical electronics and bioinformatics},
	fieldofstudy-czech = {Lékařská elektronika a bioinfomatika},
	keywords-czech = {slovo, klíč},
	keywords-english = {word, key},
	day = 25,
	month = 4,
	year = 2023,
	specification-file = {zadani_prace.pdf},
%	front-specification = true,
%	front-list-of-figures = false,
%	front-list-of-tables = false,
%	monochrome = true,
%	layout-short = true,
}

\ctuprocess

\begin{thanks}
\end{thanks}

\begin{declaration}
Prohlašuji, že jsem předloženou práci vypracoval samostatně a že jsem uvedl veškeré použité informační zdroje v souladu 
s Metodickým pokynem o dodržování etických principů při přípravě vysokoškolských závěrečných prací. 



V Praze, \ctufield{day}.~\monthinlanguage{title}~\ctufield{year}
\end{declaration}

\begin{abstract-english}
We develop \ldots
\end{abstract-english}

\begin{abstract-czech}
Rozvíjíme \ldots
\end{abstract-czech}

\begin{document}
\maketitle


%% BEGIN

%Intro
\chapter{Intruducion}


%BloodPressure
%!TEX ROOT=main.tex
\chapter{Hemodynamické parametry}
Tato kapitola se zaměří na popis hemodynamických parametrů krevního řečiště.
\section{Krevní tlak}
Krevní tlak je veličina, která vyjadřuje velikost síly proudící krve, působící na stěnu cévy. Velikost krevního tlaku závisí na síle kontrakce srdce, množstvý krve uvnitř těla,
odporu cév a elasticitě stěny cév. Krevní tlak se uvadí ve dvou typech. Systolický tlak, který vyjadřuje působící sílu krve proti stěně artérie, při kontrakci srdce a diastolický tlak vyjadřuje
tlak, kdy srdce je v klidu mezi jeho činnosti. Krevní tlak se uvádí v jednotkách $mmHg$ milimetrů rtuti, co odpovídá $133.32
    \  Pa$. Jako normální lidský krevní tlak se považuje méně jak $120 \ mmHg$ systolického a měně jak  $80 \ mmHg$ diastolického tlaku.
\subsection{Centrální aortální tlak}
Centrální aortální tlak je tlak v aortě, do které krev putuje při kontrakci srdce. \par
Centrální aortální tlak je možno měřit neinvazivním způsobem, připevněním manžety na horní část ruky nebo zápěstí. Ze snímané tlakové křivky je
možné estimovat centrální aortální tlak.

\subsection{Střední arteriální tlak}
Střední arteriální tlak (MAP) je průměrná hodnota krevního tlaku v jednom srdečním cyklu.
\begin{figure}[H]
    \caption{Průběh tlakové křivky pro výpočet středního arteriálního tlaku}
    \includegraphics[width=1\textwidth]{pictures/map.jpg}
\end{figure}
Rovnice pro přesný výpočet je
\begin{equation}
    MAP = \frac{1}{T} \int_{t}^{t + T} BP(\tau) d\tau
\end{equation}
Rovnice pro aproximaci je
\begin{equation}
    MAP = DP + \frac{1}{3}(SP - DP)
\end{equation}
Kde $DP$ je diastolický tlak a $SP$ je systolický tlak.
\section{Metody měření krevního tlaku}
Metody měření tlaku můžou být invazivní nebo neinvazivní, manuální čí automatizované. Jedním k nejvíce používaných metod pro neinvazivní měření
krevního tlaku patří auskultační metoda, která používá rtuťového sphygmomanometeru a stetoskopu pro poslouchání krevní vlny. Další nejvíce používaná metoda je oscilometrická metoda.
\subsection{Oscilometrická metoda}
Oscilometrická metodu měření tlaku využívá navrhovaný systém CarDi. Metoda spočívá v měření objemové pulzace v tepnách přenášející se přes manžetu do přístroje, ve kterém se vyhodnocují.
Amplituda těchto pulzací je závislá na rozdílu tlaku uvnitř a vně tepny, tzv. transmurální tlak. Největší amplituda při je při nulovém transmurální tlaku, to je při hodnotě středního arteriální tlaku.

\begin{figure}[H]
    \caption{Graf oscilometrických pulzací}
    \includegraphics[width=1\textwidth]{pictures/oscilometricky_tlak.jpg}
\end{figure}
\section{Rychlost šíření pulzní vlny}
Rychlost šíření pulzní vlny (PWV) je rychlost, během systolické kontrakce srdce, při které tlaková vlna krve se progpaguje arterieremi. Parametr PWV je jeden ze základních ukazatelů arteriální
elasticity. Čím je hodnota PWV vetší, tím jsou cévy měné poddajné a výsledkem je zvětšená tuhost artérií.
\begin{figure}[H]
    \caption{Postup odražené krevní tlakové vlny v těle}
    \includegraphics[width=1\textwidth]{pictures/pwv_body.jpg}
\end{figure}
Jeden ze způsobů určení parametru PWV je poměr dvojnásobné vzdálenosti od hrudního zářezu ke stydké kosti $l$ a rozdíl času primární tlakové vlny $t_1$ a odražené tlakové vlny $t_2$.
\begin{equation} \label{eq:pwv}
    PWV = \frac{2l}{t_2 - t_1}
\end{equation}
\begin{figure}[H]
    \caption{Krevní tlaková vlna s vyznačenými parametry pro výpočet parametru PWV.}
    \includegraphics[width=1\textwidth]{pictures/pwv_pressure_wave.jpg}
\end{figure}
\section{Rešerše přístrojů pro měření hemodynamických parametrů krevního řečiště}
Tato sekce se zaměří na validované systémy na trhu pro neivazivní měření hemodynamických parametrů krevního řečiště a to zejména na měření parametru PWV.

\subsection{SphygmoCor XCEL PWA/PWV}
SphygmoCor XCEL PWA/PWV je automatický systém pro analýzu krevní tlakové vlny pomocí pažní manžety.

\begin{figure}[H]
    \caption{SphygmoCor XCEL PWA/PWV}
    \includegraphics[width=0.5\textwidth]{pictures/XCEL_System.jpg}
\end{figure}
Mezi měřené parametry patří centrální aortální tlak, augmentační index a PWV. Pro měření parametru PWV je potřeba přidaná manžeta na stehno.
\par
Ovládání přístroje je pomocí připojeného osobního počítače přes USB s nainstalovaným softwarem. Software spouští terapii a následně dělá i vyhodnocení naměřených hodnot. Samostatný přístroj není schopný provádět terapii bez osobního počítače a dedikovaného softwaru.
\par
Systém SphygmoCor XCEL je validován pro použití a prodej v Evropské unii a USA. Patří do třídy II.
\subsection{Uscom BP+}
Uscom BP+ je automatizový systém pro měření centrálního krevního tlaku, augmentačního indexu a analýzy křívky krevního tlaku a dalších parametrů.
\begin{figure}[H]
    \caption{Uscom BP+}
    \includegraphics[width=0.6\textwidth]{pictures/uscom_bp.jpg}
\end{figure}
Systém BP+ provádí terapii a analýzu naměřených hodnot v jednom systému tj. bez potřeby nadřazeného systému.

\subsection{Arteriograph}
Arteriograph (Tensiomed) je systém pro neinvazivní měření hemodynamických parametrů krevního řečístě oscilometrickou metodou a pomocí jedné pažní manžety.
Metoda měření oscilometrickou metodou je patentována (US Pat. No. 20070106162) a validována invazivně.
\begin{figure}[H]
    \caption{Tensiomed Arteriograph}
    \includegraphics[width=0.6\textwidth]{pictures/arteriograph.jpg}
\end{figure}
Arteriograph může fungovat samostatně pro měření pouze krevního tlaku. Přístup k měření dalších hemodynamických parametrů je potřeba systém připojit k nadřazenému systému pomocí bluetooth s nainstalovaným specializovaným softwarem od Tensiomed.
Po připojení k nadřazenému systému, Arteriograph zasílá během terapie v reálném čase, kde se zpracují naměřené výsledky a zobrází.
%Hardware
%!TEX ROOT=main.tex
\chapter{Hardware}
Zde bude celkovy popis HW vcetne blokovych schemat
zde bude poppis Pouziteho MCU

\section{Krevní tlak a oscilometrické pulzace}

\section{Řídící jednotka}

\section{Vzduchové ventily}
Ventily jsou důležitou součástí pneumatického systému. Starají se o správný průběh terapie a také o bezpečí pacienta. \par
V systému rozlišujeme dva druhy vzduchových ventilů, uzavírací a vypouštějcí regulační. Uzavíraci ventil slouží pro oddělení manžety a pumpy. Vypouštěcí regulační ventily jsou na obou větvých pneumatického systému. Slouží jako pro regulaci tlaku v systému během terapie a také jako nouzové vypouštěcí ventily. \par
Všechny použité ventily nesmí v uzavřeném stavu propustit vzduch při tlaku $300 \ mmHg$, jinak by hrozilo nepřesné výsledky při měření a tím pádem špatná terapie.

\subsection{Uzavírací ventil}
Uzavírací ventil je důležitou součástí systému. Pneumatický systém rozdělí na dvě větve, kde jedna je část s manžetou a druhá větěv je jako referenční. \par

\begin{figure}[H]
\includegraphics[width=0.9\linewidth]{pictures/closing_valve.jpg}
\caption{Fotka uzavíracího ventilu CJAV08-2B05A1}
\label{fig:closing_valve}
\end{figure}

Pro tento účel je použit ventil CONJOIN CJAV08-2B05A1. Je to řízený napětím, normálně zavřený, vzduchový ventil typu solenoid  o $U = 5 \ V$ a vstupní proud o $I = 204 \ mA \pm 10\% $

\begin{figure}[H]
\includegraphics[width=0.9\linewidth]{pictures/closing_valve_driver.jpg}
\caption{Schéma zapojení uzavíracího ventilu}
\label{fig:closing_valve_driver}
\end{figure}

Pro řízení ventilu z výstupního pinu MCU je použitý NMOS tranzistor BSS138. BSS138 má spínací práh napětí $U_{GS} = 3.3 \ V$ což je přímo výstupní napětí z GPIO a maximální proud přes drain je $I_D = 0.22 \ A$. Resistor přes Gate a Source zajisti známé napětí, pokud bude vstup na Gate plovoucí. Tím se zamezí neznámé chovaní tranzistoru.

\subsection{Regulační ventil}
Regulační ventily slouží k regulaci tlaku v systému při terapii a také jako vypouštěcí ventily pro vrácení pneumatického systému na atmosférický tlak. Ventily jsou umístěné na každe větvy pneumatického systému. Během terapie je možno si zvolit jak moc vysoký průtok vzduchu je možný, tím můžeme regulovat tlak v obou větvých podle potřeby terapie. \par 

Regulační ventily jsou použité JQF4-6A/DC6V. Je to normálně otevřený lineární solenoid ventil. Maximální povolený tlak je $350mmHg$, řízený napětím $U = 6 \ V$ DC a proudový odběr je $I = 0.107 \ A$.\par

Napětí na ventilech je $5V$ i přes to, že ventily požadují napětí $6 \ V$. Sadou testů zjístilo, že momentální napětí vyhovuje naším požadavkům a únik tlaku při plném sevření nijak neovlivňuje terapii a přidáním $6 \ V$ by se akorát zvýšila komplexita systému. 



\begin{figure}[H]
\includegraphics[width=1\linewidth]{pictures/var_valves.jpg}
\caption{Schéma zapojení regulačních ventilů}
\label{fig:variable_valve_driver}
\end{figure}

\subsubsection{Zdroj proudu}
Regulační ventily jsou řízené napěťově řízeným zdrojem proudu jak je na obrázku \ref{fig:variable_valve_driver}.\par
Ventily jsou napojené na drain NMOS tranzistoru, přes který jde konstatní napětí požadované ventilem. Proud se řídí operačním zesilovačem, který má na neinvertujícím vstupu $U_+$ napojené řídící napětí $U_i$. Výstup operačního zesilovače je spojen s gate tranzistoru. Source tranzistoru je spojen s invertujícím vstupem $U_-$ operačního zesilovače a také paralelně k zemi jsou zapojené resistory $R_{||}$, které určují maximální možný proud na regulačních ventilech. Výsledný proud je:
\begin{equation}
    \label{eq:current_source}
    I = \frac{U_i}{R_{||}}
\end{equation}
 V případě na obrázku \ref{fig:variable_valve_driver} paralelní resistory $R_1 = R_2 = 51 \ \Omega$ mají výslednou hodnotu:  
\begin{align*}
    R_{||} = \frac{R_1 R_2}{R_1 + R_2} = \frac{51}{2} = 25.5 \ \Omega
\end{align*}
Maximálním napětí, které umožní MCU z GPIO pinu je $3.3V$ proto maximální možný proud na regulačních ventilech je:
\begin{align*}
    I = \frac{U_i}{R_{||}} = \frac{3.3}{25.5} \approx 129 \ mA
\end{align*}

Pokud budeme brát v úvahu ideální OZ, tak do invertujícího $U_-$ a neinvertujícího $U_+$ vstupu jde nulový proud, kde $U_+ = U_-$ a výstup z OZ je 
\begin{align}
    U_o = A(U_+ - U_-)
\end{align}
kde $A [-] $ je zesilovační činitel, který se blíží k nekonečnu. Pokud bude na výstupu OZ nulové napětí, tranzistor je uzavřen a napětí na source je také nulové. Pokud ale například dáme řídící napětí třeba na $U_i = 1V$, poté se OZ bude snažit, aby rozdíl $U_+ - U_- = 0$, tak na výstupu OZ se bude zvyšovat napětí dokud napětí na source nebude $U_- = U_i$. To znamená, že přes regulační ventily právě bude procházet proud z rovnice \ref{eq:current_source}. 

Použitý NMOS tranzistor je BSS138, který má minimální práhové napětí $U_{GS(th)} = 0.5 \ V$, to je napětí, při kterém začne protékat proud. To znamená, že minimální řídící napětí musí být $U_{i} = 0.5 \ V$


\subsubsection{Řídící signál}
Řídící signál je čtvercový pro řízení proudového zdroje. PWM signál z MCU o frekvenci $f_{PWM} = 25 kHz$ je filtrován pomocí RC článku o zlomové frekvenci $f_c = 159 Hz$, který slouží pro modulaci řídícího PWM signálu na konstatní napětí. 
\par
Pulse Width Modulated(PWM) signál je periodický čtvercový signál s fixní periodou a měnící se poměrem času v log.1 a log.0, také nazývané jako střída(Duty Cycle). Průměrné napětí PWM signálu je
\begin{equation}
    U_{out} = U_{max} \cdot Duty Cycle
\end{equation}
kde $U_{max}$ je maximální amplituda PWM signálu. \par


Pomocí fourierovy analýzy PWM signálu můžeme vidět, že PWM signál se neskládá pouze z jedné frekvence, ale z mnoha (Obrázek \ref{fig:pwm_spectrum}). 

\begin{figure}[H]
    \centering
    \includegraphics[width=1\linewidth]{pictures/pwm_spectrum_microchip90003250A.jpg}
    \caption{Spektrum PWM signálu převzatého od Microchip TB3250 kde $f_{PWM}$ je frekvence PWM signálu a $T$ je jeho perioda.}
    \label{fig:pwm_spectrum}
\end{figure}

Největší amplitudu typyckého PWM signálu má na její nastavené frekvenci $f_{PWM}$ a ostatní harmonické frekvence jsou její celočíselné násobky. Tyto frekvence přidávají nechtěný šum a můžou být potlačeny pomocí filtru typu dolní propust. 


\begin{figure}[H]
    \centering
    \includegraphics[width=1\linewidth]{pictures/rc_pwm_spectrum_microchip90003250A.jpg}
    \caption{Požadované odstraněné frekvence ve spektru PWM signálu převzatého od Microchip TB3250 kde $f_{PWM}$ je frekvence PWM signálu a $T$ je jeho perioda, $f_{c}$ je zlomová frekvence filtru.}
    \label{fig:unwanted_pwm_spectrum}
\end{figure}

Použitá dolní propust je RC filtr. Podle obrázku \ref{fig:variable_valve_driver} RC filtr je složený z odporu $R = 10 \ k\Omega$ a kondenzátoru $C = 100 \ nF$ kde výstupní napětí je napětí na kondenzátoru. 
\begin{equation}
    U_C(j\omega) = U_{in}(j\omega) \frac{1}{1 + j\frac{\omega}{\omega_0}}
\end{equation}
\begin{equation}
    \omega_0 = \frac{1}{RC} \Rightarrow f_c = \frac{\omega_0}{2\pi} = \frac{1}{2\pi RC} 
\end{equation}
Kde $f_c = 159 \ Hz$ je zlomová frekvence filtru.

\begin{figure}[H]
    \centering
    \includegraphics[width=1\linewidth]{pictures/var_rc_filter.png}
    \caption{Frekveční charakteristika použitého RC filtru. Obrázek je poskytnut z webové stránky \url{http://sim.okawa-denshi.jp/en/CRtool.php}}
    \label{fig:var_rc_filter_char}
\end{figure}

Na obrázku \ref{fig:var_rc_filter_char} můžeme vidět, že signál se začne atenuovat na zlomové frekvenci a fáze signálu se začne posouvat na frekvenci $\frac{f_c}{10}$.  \par

Aby výstupní řídící signál byl co nejvíce konstatní musíme zvolit vstupní frekvenci PWM signálu co nejvyšší, aby harmonické složky byly co nejvíce utlumeny. 

\begin{figure}[H]
    \centering
    \includegraphics[width=1\linewidth]{pictures/filtered_pwm.jpg}
    \caption{Simulace RC filtru při vstupním PWM signálu o střídě $50\%$ a frekvencí $f_{PWM} = 1000 \ Hz $ (zelená) a $f_{PWM} = 25000  \ Hz $ (modrá). Simulace byla provedena v programu LTspice XVII.}
    \label{fig:filtered_pwm}
\end{figure}
Rušení výstupního signálu ovlivní chování regulačního ventilu. Toto rušení způsobí periodickou změnu amplitudy výstupního signálu a regulační ventil se podle této amplitudy bude periodicky otevírat a zavírat. Při zvolené frekvenci PWM $f_{PWM} = 25000 \ Hz $ je změna napětí $\approx 30 \ mV$. To způsobý změnu proudu $I = \frac{0.030}{25.5} = 11 \ mA $. \par

Volba velikosti prvků RC článku ovlivní schopnost reakce na změnu vstupního napětí. Výstupní napětí RC článku je 
\begin{equation}
    U_{C}(t) = U_{in}(t)(1 - e^{-\frac{t}{RC}}) = U_{in}(t)(1 - e^{-\frac{t}{\tau}})
\end{equation}
Časová konstanta $\tau = RC = 10 \times 10^{3} \cdot 100 \times 10^{-9}= 1\ ms$ definuje čas, který potrvá aby napětí na kondenzátoru dosáhlo $U_{C}(\tau) = U_{in}(\tau)(1 - e^{-1})$, což je $\approx 63 \ \% $ vstupního napětí $U_{in}$. 





\section{Digitalizace analogových signálů}



\section{Napájení}
Vstupní napájení je použito pro napájení celého přistroje. Vstupní napětí je $U_{in} = 5V DC$, které poskytuje napájení pro všechny součástky na přístroji. Vstupní napětí je poté pomocí regulátoru napětí s nízkým úbytkem usměrněno na $3.3 V$ pro napájení MCU, sensorů a ostatních komponentů.

\begin{figure}[H]
\includegraphics[width=0.9\linewidth]{pictures/power.jpg}
\caption{Schéma zapojení vstupního napájení}
\label{fig:power_input}
\end{figure}

Celkový proudový odběr přístroje je ..... 

Přístroj je opatřen $4 [A]$ pojistkou a ochranou proti opačné polaritě.
Ochrana proti opačné polaritě zajistí při špatném zapojení, aby proud neprotékal přístrojem, ale musí se zajistit, aby ztrátový výkon 
\begin{align*}
  W = I^2 R  
\end{align*}

byl co nejmešní při správném zapojení. Proto je použit PMOS tranzistor jako ochrana obvodu, který . Gate tranzistoru je připojena k zemi a mezi Drain a Source protéká proud při správném zapojení napájecího zdroje. Protože $ U_G = 0 [V]$ a $U_S = U_{in}$, tak 
\begin{align*}
U_{GS} = U_G - U_S
\end{align*}
$U_{GS} = -U_{in} $, proto je potřeba, aby 
\begin{align*}
 U_{GS(ON)} > -U_{in}   
\end{align*}
Při opačném zapojení napájení $U_S = -U_{in}$ a $ U_G = 0 V$, tak  $U_{GS} = U_{in}$ tranzistor je vypnut a přes obvod neprotéká proud.
V návrhu je použit tranzistor TPM9305PS3, který má  $U_{GS(ON)} = -2.5V $,  $I_D = -4.1A$ a $R_{DS(ON)} = 52m \Omega$ při $U_{GS} = -4.5V$. 
Ztrátový výkon bude 
\begin{align*}
  W_{loss} = I^2 R \approx (3)^2 (0.053) =  159 mW
\end{align*}

Na obrázku \ref{fig:power_input} je ještě připojena mezi $U_G$ a $U_S$ zenerova dioda, která zamezí maximální napětí, pro ochranu tranzistoru. Pokud napájecí zdroj bude mít větší napětí něž maximální povolené napětí na $U_{GS}$, zenerova dioda upne $U_{GS}$ na její maximální napětí.
\par
Pro maximální zamezení rušivých jevů a braní ohledu na EMC jsou připojeny paralelně dva blokovací kondenzátory.

\begin{figure}[H]
\includegraphics[width=0.9\linewidth]{pictures/ldo_3v3.jpg}
\caption{Schéma zapojení regulátoru napětí z 5V na 3.3V}
\label{fig:stepdown}
\end{figure}
Na obrázku \ref{fig:stepdown} je schéma zapojení lineárního regulátoru napětí s nízkým úbytkem AZ1083-3.3. Vstupní napětí je v rozmezí $1.5V \leq U_{in} \leq 12V $. Výstup regulátoru je fixní na $U_{out} = 3.3V$ a maximální výstupní proud je $I_{out(MAX)} = 5A$. Zapojení regulátoru je podle doporučeného zapojení v datasheet.


\section{Senzory}
Tato sekce se zaměří na popis a použití senzorů a to zejména tlakových. Tlakové senzory tvoří nezbytnou část celkového přistroje a rozhodují o celkovém konfortu pacienta a také o přesnost výsledné terapie. \par
Parametry senzorů tlaku vychází z parametrů terapie. Pneumatický systém může být pod tlakem až $300 \ mmHg = 40 \ kPa$, tento požadavek musí splňovat všechny senzory napojené do pneumatického systému.

\subsection{Senzor tlaku}
Senzor tlaku se používá na snímání tlaku v jednotlivých větvý pneumatického systému. \par


Použité sensory tlaku jsou NPX MP3V5050GC6U. 

\begin{figure}[H]
    \centering
    \includegraphics{pictures/nxp_sensor.jpg}
    \caption{Senzor tlaku NPX MP3V5050GC6U}
    \label{fig:nxp}
\end{figure}

Je to analogový sensor tlaku od firmy NXP ze série peizorezistivních převodníků. Parametry jsou následovné: 
\begin{center}
\begin{tabular}{lllllll}
\toprule
Charakteristika                        & Symbol &  Min  & Typ & Max   & Jednotka &  \\ \midrule
Rozsah tlaku                           & $P$    &  0    &   -   & 50   & $kPa$ & \\
Vstupní napětí                         & $U_{s}$& - & 3.3 &  -  & $V$ &  \\
Vstupní proud                          & $I_{s}$ & - & 10  &  -  & $mA$ &  \\
Napěťový offset($0^{\circ}$ až $ 85^{\circ} $) & $U_{off}$& - & 0.188& -  & $V$ &  \\ 
Full Scale Output (4)                  & $U_{FSO}$ & &2.77& & $V$ & \\
Přesnost($0^{\circ}$ až $ 85^{\circ}$) & - & - & - & $\pm 2.5 $  &  $\%$ &\\
Citlivost                              & $\frac{U}{P}$ & - & 54 & - & $\frac{mV}{kPa}$ & \\
\bottomrule
\end{tabular}
(4)-Maximální napětí při největším hodnoceném tlaku.
\end{center}

Zapojení senzoru je na separátní DPS podle doporučeného schématu \ref{fig:nxp_recommended} z datasheetu. 


\begin{figure}[H]
    \centering
    \includegraphics{pictures/nxp_recommended.jpg}
    \caption{Doporučené schéma zapojení senzoru tlaku NPX MP3V5050GC6U. Kde $V_S$ je vstupní napájecí napětí a $V_{out}$ je výstupní napětí.}
    \label{fig:nxp_recommended}
\end{figure}

Analogový výstup ze sensoru je připojen na interní AD převodník MCU. 

\subsubsection{Převodní charakteristika}
Převodní charakteristika výstupního napětí $U_{o} \ V$ na tlak $P \ kPa$je 
\begin{equation}
    P = \frac{U_o \pm ERROR}{0.018 \cdot U_s} - \frac{0.04}{0.018}
    \label{eq:nxp_transfer}
\end{equation}

\begin{figure}[H]
    \centering
    \includegraphics{pictures/nxp_transfer.jpg}
    \caption{}
    \label{fig:nxp_transfer}
\end{figure}


\subsection{Diferenční sensor tlaku}
Diferenční sensor tlaku slouží na snímání malých tlakových pulzací. Porovnává tlak mezí první a druhou (referenční) větvý systému. Po natlakování pneumatického systému až na $300 \ mmHg$ uzavírací ventil oddělí systém na dvě větve. Rozdíl tlaků ve větvy může být $300 \ mmHg$ neboli $40 \ kPa$. \par

Diferenční sensor tlaku byl zvolen Amphenol ELVH-L02D-HRRD-C-NAA4. Je to analogový senzor tlaku určený na snímání ultra nízký tlaků. 


\begin{figure}[H]
    \centering
    \includegraphics{pictures/amphenol.jpg}
    \caption{Diferenční sensor tlaku Amphenol ELVH-L02D-HRRD-C-NAA4}
    \label{fig:amphenol}
\end{figure}
Tato série umožňuje snímat rozdíl tlaků v rozsahu $\pm 0.5 \ kPa$ a na každý vzduchový port je možno aplikovat tlak o maximální hodnotě $67 \ kPa$. \par 
Přesnosti snímaného tlaku je  $0.25 \ \% FSS$, kde FSS(Full Scale Span) je algebraický rozdíl napětí při nejmenším možném specifikovaném tlaku a při maximálním specifikovaném tlaku $FSS = \pm 1.32 \ V $. Napěťový offset je $U_{offset} = 1.65 V$. Z toho vyplývá, že výstupní napěťový rozsah je $0.33 \ V \leq U_o \leq 2.97 \ V$. 
Sensor je napájen ze $3.3 \ V $ a jeho proudový odběr je $2.8 \ mA $.

%Software
%!TEX ROOT=main.tex
\chapter{Software}
Tato kapitola se zaměří na popis vývojového prostředí, používaného programovacího jazyka a knihoven.
\begin{figure}[H]
    \label{fig:sw_diagram}
    \caption{Digram připojených periférii k MCU}
    \includegraphics[width=0.9\textwidth]{pictures/sw_diagram.png}
\end{figure}
MCU STM32F407ZG6 je postaveno na architektůře Arm Cortex M4 s přidaným jádrem a instrukcemi pro výpočty plouvoucích čísel.
Programování MCU probíha v programovacím prostředí od ST Microeletronics STM32CubeIDE, které má v sobě zabudovaný kompilátor, prostředí pro debuggina a prostředky pro nahrání SW do MCU.
\begin{figure}[H]
    \caption{Ukázka vývojového prostředí STM32CubeIDE pro mikroprocesory STM32}
    \includegraphics[width=0.9\textwidth]{pictures/cubeide.jpg}
\end{figure}
STM32CubeIDE také sprostředkovává ovladače pro komunikaci s internímy periferiemy a prostředí pro konfiguraci MCU.
\begin{figure}[H]
    \caption{Konfigurace GPIO pinů MCU}
    \includegraphics[width=0.9\textwidth]{pictures/mcu_settings.jpg}
\end{figure}
Po zvolení konfigurace MCU STM32CubeIDE samo vygeneruje základní softwarovou inicializaci periférií.
\par
Nahrání SW a debuggování MCU musí být provedeno přes programátor od firmy ST Microeletronics ST-LINK.
\begin{figure}[H]
    \caption{Programátor ST-LINK V2/ISOL}
    \includegraphics[width=0.5\textwidth]{pictures/stlink.jpg}
\end{figure}
MCU využívá komunikační protokol Serial Wire Debug vyvinutý přímo firmou ST M. Pro připojení k DPS je použitý speciální kabel od firmy TagConnect TCP2030, který se připojí na kontakt na povrchu DPS.
\begin{figure}[H]
    \caption{Programovací kabel TagConnect TCP2030}
    \includegraphics[width=0.6\textwidth]{pictures/tcp2030.jpg}
\end{figure}

%Results
%!TEX ROOT=main.tex
\chapter{Výsledky}
%Conclusion


%% END

\chapter{Závěr}

\appendix

%\printindex


\bibliographystyle{amsalpha}
%\bibliography{ctutest}

\ctutemplate{specification.as.chapter}

\end{document}